%!TEX program = xelatex
% 完整编译: xelatex -> bibtex -> xelatex -> xelatex
\documentclass[lang=cn,11pt,a4paper]{elegantpaper}
\lstset{,style=bash}

\title{定位与导航实验报告}
\author{Wugang Meng}
\institute{哈尔滨工业大学(威海)}

\version{0.1}
\date{\zhtoday}

\begin{document}

\maketitle

\begin{abstract}
本文为《**智能感知系统》项目中的定位与建图实验、自主导航实验给出了实验测试大纲。介绍了测试环境,测试软件硬件;具体化了实验步骤和实验指标,并体现了实验结果。
\keywords{定位与建图,自主导航,试验大纲}
\end{abstract}



\section{测试概要}

\subsection{测试目标}
进行本实验的目的是验证毫米波传感器能否与机器人操作系统(ROS)环境中流行的地图和导航库一起使用,这是许多机器人技术人员所熟悉的。
本实验将Octomap服务器和move\_base库与TI的mmWave ROS驱动程序软件包以及DJI RoboMaster底盘库一起使用,以与TI mmWave传感器接口。本实验支持使用IWR1443ISK ES1.0 EVM或IWR1443ISK ES2.0 EVM。借助TI驱动程序和\href{ros.org}{ROS社区}的软件,工程师可以快速,轻松地评估机器人导航和避开物体。
本实验室采用四个传感器一起使用。通过使用四个传感器,该机器人可以具有$360^{\circ}$视野,因此该机器人能够检测周围的物体,以便更好地进行地图绘制和导航。

\subsection{测试环境}
\subsubsection{场地环境}
本实验采用DJI RMUC 的核心比赛场地,也被称为“战场”。如图所示\ref{fig:map}战场是一个长为 28 米、宽为 15 米的区域,内部为木质结构,表
面贴地胶(厚度 3mm),主要包含基地区、高地区、资源岛区、补给区和飞行区等。战场外围有上边沿距离战场地面高度为 2.4 米的黑色钢制围挡。
全文描述的所有场地道具的尺寸误差均在$±5\%$以内。场地说明图纸尺寸参数单位为mm。

\begin{figure}[htbp]
  \centering
  \includegraphics[width=\textwidth]{map.png}
  \caption{测试场地渲染图}
  \label{fig:map}
\end{figure}

\subsubsection{硬件设备}
移动作战平台基于模块化设计,分为动力模块,感知模块,控制模块和计算模块。所有模块支持快拆结构,各模块可单独编程与调试使用。
\begin{figure}[htbp]
  \centering
  \includegraphics[width=\textwidth]{chassis.png}
  \caption{动力模块设计图}
  \label{fig:chassis}
\end{figure}

\begin{figure}
  \centering
  \subcaptionbox{直流无刷减速电机\label{fig:3508}}
  {\includegraphics{m3508.png}}
  \subcaptionbox{电子调速器\label{fig:620}}
  {\includegraphics{C620.png}}
  \caption{电机与电子调速器设计图}\label{ele}
\end{figure}

\begin{figure}[htbp]
  \centering
  \includegraphics[width=\textwidth]{manifold.png}
  \caption{计算模块}
  \label{fig:manifold}
\end{figure}

\begin{enumerate}[label=\arabic*).]
  \item \textit{动力模块}\\
      动力模块是承载和安装机器人动力系统及其附属部件的机构。如图\ref{fig:chassis}所示使用麦克纳姆轮,支持全向运动。动力系统由RoboMaster M3508 P19 无刷直流减速电机(图\ref{fig:3508})和RoboMaster C620 电子调速器组成(图\ref{fig:620})。
      C620无刷电机调速器采用32位定制电机驱动芯片,配合磁场定向控制(FOC)技术,实现对电机转矩的精确控制。M3508直流无刷减速电机是专为中小型移动平台和机器人等量身打造的高性能伺服电机,可搭配C620电调实现正弦驱动,相比传统方波驱动具有更高的效率、机动性和稳定性。本产品减速箱减速比约为19:1。
  \item \textit{感知模块}\\
      感知模块应组装一个远距高分辨级联模组mmWave-CAS EVM如图\ref{fig:CAS}和四个mmWave EVM如图\ref{fig:mini}所示。根据EVM的版本,安装可能会略有不同。
      在所示示例中,用于安装mmWave EVM的支架是3D打印的。
      对于本演示而言,至关重要的是,必须每隔90度将mmWave EVM安装在机器人周围,与机器人中心等距。
      在所示示例中,将12V至5V转换器安装在顶板中心下方,而USB分配器则放置在顶板下方。四个EVM使用USB分配器连接到计算模块。
  \item \textit{控制模块}\\
      控制模块使用RoboMaster 开发板C型 (STM32F427) 作为主控板,开发板主控芯片为STM32F427IIH6,拥有丰富的扩展接口和通信接口包括12V/5v/3.3v电源接口,CAN接口,UART接口、可变PWM接口、SWD接口,
      板载IMU传感器,可配合RoboMaster出品的M3508、 M2006直流无刷减速电机、UWB模块以及计算模块等产品使用,亦可配合DJI飞控SDK使用,配件丰富。
  \item \textit{计算模块}\\
  计算模块使用的是英特尔酷睿 i7-8550U 处理器,功率范围5-60W,具备优秀的处理能力和响应速度的同时仅有200g左右的超轻重量。
\end{enumerate}


\begin{figure}
  \centering
  \includegraphics[width=\textwidth]{radar_b.png}
  \caption{mmWave-CAS EVM}\label{fig:CAS}
\end{figure}

\begin{figure}
  \centering
  \includegraphics[width=\textwidth]{radar_mb.png}
  \caption{mmWave EVM}\label{fig:mini}
\end{figure}
  

\begin{table}[!htbp]
  \centering
  \caption{计算模块规格参数}
  \begin{tabular}{cc}
    \toprule
    参数名称  & 参考数值  \\
    \midrule
    重量  & 205g   \\
    尺寸 & $91 \times 61 \times 35$ mm  \\
    处理器 &  i7-8550U \\
    内存 & 8GB 64bit,DDR4 2400MHz \\
    SATA-SSD & 256GB\\
    网络 & 千兆以太RJ-45接口\\
    USB & USB3.0(Type A)$\times 2$,USB3.0(Micro-B)$\times 1$ \\
    I/O & UART $\times 1$  \\
    功率 & 5-60w \\
    电源 & 15.2-27.0V电源接口$\times 2$  \\
    工作温度 & -25至45$^{\circ}$C\\
  \bottomrule
  \end{tabular}%
\end{table}%

\subsubsection{软件平台}

软件平台以机器人传感器$\rightarrow$感知$\rightarrow$决策$\rightarrow$规划$\rightarrow$控制$\rightarrow$执行器 的环路进行架构,不同模块具体以ROS Package的形式维护,模块和其数据流如下图所示。
\begin{figure}
  \centering
  \includegraphics[width=\textwidth]{flow.png}
  \caption{架构与模块介绍}\label{fig:flow}
\end{figure}

\begin{enumerate}[label=\arabic*).]
  \item \textit{传感器、控制器与执行器}\\
      中心模块集成传感器模块(雷达、相机、IMU等)、嵌入式控制平台(执行实时任务,如闭环控制和数据采集与预处理)与执行器(电机等),负责sensing和control两大任务,具体ROS Package为包含嵌入式SDK的roborts\_base,相机的roborts\_camera以及相关传感器驱动包。
  \item \textit{感知部分}\\
      感知部分包括机器人定位、地图的维护和抽象、目标识别与追踪等。localization模块负责机器人定位,是导航系统中强依赖的节点,其主要通过获得一系列的传感器信息,通过特定算法的处理,最终得到机器人坐标系到地图坐标系的变换关系,也就是机器人在地图中的位姿。
      map模块负责机器人地图维护,目前采用ROS开源Package map\_server。
      costmap模块负责代价地图维护,集成了静态地图层,障碍物层和膨胀层,主要用于运动规划部分,不单纯针对规划使用。
      detection模块负责目标识别和追踪,主要利用mmWave-CAS EVM对拍摄到的点云信息进行物体的聚类与跟踪。
  \item \textit{任务调度与决策部分}\\
  任务调度与决策部分包括调度感知输入模块和调度规划执行输出模块的接口,以及决策的核心框架。
  decision模块为机器人决策框架,以行为树(BehaviorTree)为决策框架。blackboard模块调度各种模块的感知任务获取信息,behavior模块集成了离散动作空间的各种动作或行为。
  executor模块是behavior模块的依赖,其包含底盘和云台内不同模块内不同抽象程度的机器人任务委托接口(例如调度底盘运动规划执行)。
  \item \textit{运动规划部分}\\
  运动规划部分是运动规划功能模块,由决策部分中chassis\_executor模块来调度完成导航。global\_planner模块负责机器人的全局路径规划,全局路径规划(简称全局规划)是导航系统中运动规划的第一个步骤,在给定目标位置后,根据感知的全局代价地图搜索得到一条无碰撞的最短路径(即一系列离散的坐标点),然后作为输入传递给局部轨迹规划模块控制机器人的具体运动。全局路径规划A Star算法,
  是一种静态路网中求解最短路最有效的方法。公式表示为:
  $$
    f(n)=g(n)+h(n)
    \label{a*}
  $$
  其中$f(n)$是节点$n$从初始点到目标点的估价函数,$g(n)$是在状态空间中从初始节点到$n$节点的实际代价,$h(n)$是从$n$到目标节点最佳路径的估计代价。在实际的全局路径规划中,静态网路由全局代价地图提供。
  local\_planner模块负责机器人的局部轨迹规划模块,局部路径规划模块根据机器人的里程计信息,雷达的感知信息,结合全局路径规划给出的最优路径,计算出机器人当前能够避免与障碍物碰撞的最优速度。局部路径规划的相关算法参考 Time Elastic Band,算法建立了轨迹执行时间的目标方程,在满足与障碍物保持安全距离, 运动时遵守动力学约束的前提下,优化机器人实际运行轨迹,最终求出机器人可执行的速度。
\end{enumerate}



\subsection{测试方案}
在重量和尺寸检测方面,分别称量设备的重量和尺寸。在功耗测量方面,通过稳压限流模块精准控制单一设备功耗。在建图精度测试方面,通过定位时间戳与室内定位基站作为参考,比较测量误差。
在定位精度方面,在图\ref{fig:map}场地进行测试,通过小车车载LPS定位装置与UWB定位装置,完成规定路径的行驶,通过比较UWB与LPS的定位航迹,计算LPS的定位误差。

\subsection{条件与限制}
本次实验等比例模拟了某区域的城市环境、地面材料及战车尺寸、质量等参数,力图复现一个真实且残酷的战场环境。
但仍受以下客观条件限制。
\begin{enumerate}
  \item  本实验在室内封闭进行,因而无法使用无人机设备进行空中拍摄,故仅提供侧轴视角和POV视角测试视频;
  \item  因为UWB、LPS、IMU和里程计都存在误差,因此定位与建图结果在此误差基础上会存在一定且无法消除的偏移;
  \item  高分辨识别雷达(图\ref{fig:CAS})和建图与导航雷达(图\ref{fig:mini})采用解耦合设计,高分辨识别雷达仅作目标识别之用不参与建图与导航过程;
\end{enumerate}

\section{测试计划}

\subsection{测试项目}
完成设备重量检测、功耗测量、建图精度测试、定位精度测试等内容,为后续实验提供指标参数与参考依据。
测试目标包括高分辨雷达,建图雷达,导航基准为UWB接收机。

\subsection{测试技术流程}

\subsubsection{硬件指标测量}
直接对设备测量即可。测量项目包括尺寸,质量,设备功耗等。

\subsubsection{软件指标测试}
首先连接STM32设备的虚拟串口,lsusb可以查看Vendor和Product的ID,然后创建并配置/etc/udev/rules.d/roborts.rules文件:
\begin{lstlisting}[language=bash]
  KERNEL=="ttyACM*", ATTRS{idVendor}=="0483", ATTRS{idProduct}=="5740", 
  MODE:="0777", SYMLINK+="serial_sdk"
\end{lstlisting}
同理配置雷达。然后重新加载并启动udev服务,可能需要重新插拔设备后生效。
\begin{lstlisting}[language=bash]
  SUBSYSTEM=="tty", ATTRS{idVendor}=="10c4", ATTRS{idProduct}=="ea70",
  SYMLINK+="mmWave_%s{serial}_%E{ID_USB_INTERFACE_NUM}"
\end{lstlisting}


\subsubsection{环境建图指标测量}
首先如图3-2中所示,安装LPS收发天线,并如图3-3所示,做好防水处理,并在图3-1所示场地中摆放六组天线,
通过线缆与射频装置连接;其次,通过图3-4所示,利用全站仪在中心点对各个雷达天线位置进行精准定位,
将定位值输入电脑中;最后,开启LPS,通过一段时间的通信与同步,
开始推动小车按照路线行驶,待行驶结束,完成实验,计算定位误差。

\subsubsection{自主导航指标测量}

\subsection{测试进度安排}


\section{测试结果}
截止到 2019 年 10 月 17 日,ElegantPaper v0.08 版本发布,ElegantPaper 模板在 Github 上的收藏数(star)达到了 164。在此特别感谢 China\TeX{} 以及 \href{http://www.latexstudio.net/}{\LaTeX{} 工作室}对于本系列模板的大力宣传与推广。

如果你喜欢我们的模板,你可以在 Github 上收藏(Star)我们的模板。
\begin{figure}[htbp]
  \centering
  \includegraphics[width=\textwidth]{star.png}
  \caption{一键三连求赞}
\end{figure}

\section{测试总结}
如果您非常喜爱我们的模板,你还可以选择捐赠以表达您对我们模板和我的支持!

\begin{figure}[htbp]
  \centering
  \includegraphics[width=0.5\textwidth]{donate.jpg}
\end{figure}

\textbf{赞赏费用的使用解释权归 Elegant\LaTeX{} 所有,并且不接受监督,请自愿理性打赏}。10 元以上的赞赏,我们将列入捐赠榜,谢谢各位金主!

\begin{table}[!htbp]
  \centering
  \caption{Elegant\LaTeX{} 系列模板捐赠榜}
  \begin{tabular}{crcc}
    \toprule
    捐赠者   & 金额 & 时间 & 渠道 \\
    \midrule
    Lerh  & 10 元  & 2019/05/15 & 微信 \\
    越过地平线 & 10 元    & 2019/05/15 & 微信 \\
    大熊 &  20 元 & 2019/05/27 & 微信 \\
    * 空 & 10 元 & 2019/05/30 & 微信\\
    \href{http://www.latexstudio.net/}{latexstudio.net} & 666 元 & 2019/06/05 & 支付宝\\
    Cassis & 11 元 & 2019/06/30 & 微信\\
    * 君 & 10 元 & 2019/07/23 & 微信\\
    * 萌 & 19 元 & 2019/08/28 & 微信 \\
    曲豆豆 & 10 元 & 2019/08/28 & 微信 \\
    李博 & 100 元 & 2019/10/06 & 微信\\
    Njustsll & 10 元 & 2019/10/11 & 微信 \\
  \bottomrule
  \end{tabular}%
\end{table}%

\section{常见问题 FAQ}



\section{示例}

为了让大家更加清楚最终的论文效果,如下给出两篇使用 ElegantPaper 模板排版的工作论文示例,也欢迎大家“投稿”!

\bibliography{wpref}

\end{document}
